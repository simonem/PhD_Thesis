\chapter*{Abstract}

Crises are sequences of alarming events that put communities in danger,
affecting millions of people worldwide. Marked by a mix of foreseeable
and unexpected circumstances, a crisis can ultimately lead to a
disastrous event (e.g.~an hurricane or a nuclear explosion) causing huge
damages and loss of human lives. Continuous training for preparedness
empowers crisis workers (e.g.~firefighters, paramedics, managers) to
achieve better performances and commitment in providing help to the
communities struck by a crisis. To achieve those goals the body of
research in reflective learning provide theoretical tools to guide
data-driven, collaborative reflection on work experiences towards
changes in behaviour. ICT support for reflective learning facilitates
the process by providing user interfaces for capturing work experiences,
visualising discrepancies as reflection triggers; and by supporting
sharing of learning outcomes. Yet current ICT tools don't consider the
very specific, situated nature of crisis work. While data capturing
tools lack interaction paradigms suitable for being used
\emph{in-action}, visualisation tools struggle in providing the user
with context information needed to ground reflection \emph{on-action}
and to achieve learning outcomes structured to be easily shared among
colleagues.

The research in this thesis investigates how theory in the field of
embodied and sensing-based interaction can inform the design of computer
interfaces to better assist reflection practices in the case of crisis
training. This thesis explores how conceptual tools from reflective
learning theory can be implemented in technology tools to make the
capture of work experiences lightweight and pervasive, and interaction
with reflection-useful information tangible, situated and playful.

The work is grounded on design science methodology. Six field studies
have been performed during large physical simulations of crisis work.
Exploratory studies drove the design and production of eight prototypes
of sensing-based interfaces. Software and hardware rapid prototyping
techniques, opens source and digital manufacturing tools have been
largely employed. Prototypes were eventually returned to the field and
tested against acceptance, usability and impact on learning. Results
from evaluations were used to validate existing theories and for the
development of new constructs. New application domains for the use of
the technology have been investigated during academic visits at City
London University and MIT. Commercial exploitation of research outcomes
are being discussed.

The resulting contributions add new knowledge to guide the design of
novel sensing-based interfaces to support continuous training of crisis
workers. To this end, it is demonstrated how conceptual tools from
reflective learning theory can be mapped to technology to support the
\emph{capture}, \emph{re-creation} and \emph{generation} of work
experiences. Seven challenges to drive the design of
experience-capturing tools are provided. The challenges shed light on
\emph{what} information is relevant and \emph{how} to capture relevant
information and they were explored with the production of prototypes of
wearable data capturing tools. Further, the thesis contributes with
novel paradigms derived from sensing-based interaction. The paradigms
have been implemented in embodied user interfaces to reduce distraction
while \emph{capturing} experiences at work, to allow for
\emph{re-creating} past experiences situated in a physical context that
provides prompts for reflection; and to allow \emph{generating} engaging
and collaborative work experiences by means of serious games. Building
prototypes of such user interfaces requires a wide range of competences
in software and hardware engineering. Lessons learnt from the author's
experience provide knowledge for the creation of designer's tools to
ease rapid prototyping of sensing-based interfaces.


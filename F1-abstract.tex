\chapter{Abstract}

Continuous training for preparedness empowers crisis workers (e.g. firefighters, paramedics) to achieve better performance and commitment in providing help to the communities struck by a nature or human-caused crisis. To achieve these goals the body of research in reflective learning provides theoretical tools to guide data-driven, collaborative reflection on work experience towards changes in behaviour. ICT support for reflective learning facilitates the process by providing technologies for capturing work experience, visualising discrepancies as reflection triggers; and by supporting sharing of learning outcomes. Yet current ICT tools do not consider the very specific, situated nature of crisis work. While data capturing tools lack interaction paradigms suitable for being used during work, visualisation tools struggle in providing the user with the context information needed to ground reflection on past work experiences and to achieve learning outcomes that are structured to be easily shared among colleagues.

The research in this thesis investigates how theory in the field of embodied and sensing-based interaction can inform the design of computer interfaces to better assist reflection practice in the case of crisis training. This thesis explores how conceptual tools from reflective learning theory can be implemented in technology tools to make the capture of work experience lightweight and pervasive, and interaction with reflection-useful information tangible, situated and playful.

The work is grounded on design science methodology. Six field studies have been performed during large physical simulations of crisis work. Exploratory studies drove eight design iterations of sensing-based interfaces. Software and hardware rapid prototyping techniques, open source and digital manufacturing tools have been largely employed. Prototypes were eventually returned to the field and tested against acceptance, usability and impact on learning. Results from evaluations were used to validate existing theories and for the development of new constructs. Commercial exploitation of the research outcomes are being discussed.

The resulting contributions add new knowledge to guide the design of novel sensing-based interfaces to support continuous training of crisis workers. To this end, it is demonstrated how conceptual tools from reflective learning theory can be mapped to technology to support the \emph{capture}, \emph{re-creation} and \emph{generation} of work experience. Seven challenges to drive the design of experience-capturing tools are provided. The challenges shed light on \emph{what} information is relevant and \emph{how} to capture relevant information and they were explored with the production of prototypes of wearable data capturing tools. Further, the thesis contributes with novel techniques derived from sensing-based interaction. The paradigms have been implemented in embodied user interfaces to reduce distraction while \emph{capturing} experience at work, to allow for \emph{re-creating} past experience situated in a physical context that provides prompts for reflection; and to allow \emph{generating} engaging and collaborative work experience by means of serious games. Building prototypes of such user interfaces requires a wide range of competencies in software and hardware engineering. Lessons learnt from the author's experience provide knowledge for the creation of designer's tools to ease rapid prototyping of sensing-based interfaces.

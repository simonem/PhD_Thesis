\chapter{Conclusions and future works}\label{conclusions}

\todo[inline]{Revision 3}

This thesis has focused on enhancing crisis training practices with novel sensing-based technologies to capture, re-create and generate crisis work experiences.

The main research method adopted was design science. To this respect six field studies during physical simulations of crisis work were performed. Field work and literature in computer-supported reflective learning has driven eight prototyping iterations during which fairly complex hardware/software sensing-based devices were built and evaluated both in focus groups and on the field. The work has resulted in seven paper published and five declaration of inventions filed for technology transfer.

The research questions were answered by four contributions, hereafter summarised in a set of conclusions which delineate future works. \bigskip

\emph{Contribution 1} 
\newline \rule{
\textwidth}{.1pt} \emph{The MIRROR CSRL model (Chapter \ref{csrl}) can be used to technology-enhance reflective learning practices in crisis training. Activities described in the model together with requirements derived from field studies drove the design of sensing-based interfaces to support the capture, re-creation and generation of work experiences. Eight prototypes were built, including wearable data-collection tools (P3), mobile augmented reality browsers (P1) and serious games (P4,P5). Ecologies of prototypes were successfully evaluated for their impact in supporting debriefing after physical simulations of crisis work (P2). New theories about the use of sensor data as learning content emerged (P6).}

Further work is required to map with technology the two stages in the model not addressed in this research work. As evaluated in Chapter \ref{evaluation}, the investigation in this thesis is deeply connected with the specific case of crisis work. Future works will generalise the theoretical findings and the technologies produced to other domains. \rule{
\textwidth}{.1pt} \medskip

\emph{Contribution 2} 
\newline \rule{
\textwidth}{.1pt} \emph{Capturing relevant data to feed reflection processes is challenging due to the unpredictability of relevance typical of reflection (P6). In addition, the highly dynamic nature of crisis work requires to design tools that need to be adapted to the  needs of always varying scenarios. This space of opportunities is provided to researchers as a set of challenges to the design of experience-capturing tools (P3). The challenges were explored with the development of data collection tools that can be used to support debriefing (P2). Prototypes developed feature wearable sensors and an embodied user interfaces based on mnemonic body shortcuts.}

Further work is required to validate the identified challenges with more field work and to investigate challenges not addressed in this work production of prototypes. Furthermore the relevance of the challenges to other application domain has to be investigated. \rule{
\textwidth}{.1pt} \medskip

\emph{Contribution 3} 
\newline \rule{
\textwidth}{.1pt} \emph{Novel approaches inspired by the field of tangible, embodied and embedded computing can facilitate interaction with technology to support reflection. The approaches have driven the design of interfaces to reduce distraction while interacting with capturing tools during work, to allow for browsing information in physical environments that contextualise reflection; and to provide social and engaging serious gaming experiences.}

Further work points at validating the developed approaches with the production and evaluation of new prototypes; and to further formalise interaction models and design processes.

\rule{
\textwidth}{.1pt}

\medskip

\emph{Contribution 4} 
\newline \rule{
\textwidth}{.1pt} \emph{Prototypes have a central role in design science research for the validation of theories and in the development of new methods. Yet, prototyping sensing-based interfaces require large efforts due to the wide range of skills required, including hardware and software engineering, product design and assembly. Further, technology artefacts to be tested during crisis work are to be built for higher resilience compared to the ones deployed for lab testing. Despite prototyping toolkits are available, to date an holistic tool to support the development of both software and hardware complex features couldn't be found. This has resulted in large efforts required for building prototypes and in the development of a understanding of the challenges and tools currently available.}

Future work builds on the challenges experienced with the production of prototypes for this PhD work for the conceptualisation and production of a toolkit to ease the production of complex sensing-based systems. 
\rule{
\textwidth}{.1pt}
\bigskip

In conclusion, this thesis has developed knowledge about the implementation of computer supported reflection theories into novel ICT systems and sensing-based interfaces that can produce learning outcomes. Although the investigation is limited to the characteristic case of crisis training, basis for the generalisation of theories and technologies developed have been settled. Commercial exploitation of research outcomes is being explored, research funds have been granted to this purpose (see Section \ref{exploitation-of-research-contributions}). Future works aim at generalising research findings to new domains and to commercially exploit research results. 

\chapter{The case of crisis training}\label{crisis}

In this chapter I provide an overview on crisis training, preparedness and management activities, highlighting how reflection can improve learning outcomes. These three activities constitute a waterfall process, with training exercises aiming at making workers more prepared towards managing a crisis at best.

\begin{figure}
	[h!] \centering 
	\includegraphics[width=1
	\textwidth]{3steps} \caption{Three activities to better deal with a crisis} \label{fig:three-stages} 
\end{figure}

\todo{delete it?}

\section{Crisis management}\label{crisis-management}

The terms \emph{crisis}, \emph{emergency} and \emph{disaster} are often used as synonyms. They deal with events that belong to the ``un-ness'' category: unexpected, undesirable, unimaginable, and often unmanageable situations \autocites{Boin:2007wt}{hewit}. In this space made up of both foreseeable and unexpected elements, crisis management works at anticipating what can be predicted in order to minimise the unforseen\footnote{Source: http://emergency-planning.blogspot.it}.

In this thesis the term \emph{crisis} refers to a single or sequence of problematic events that may lead to a dangerous situation, whether that is an \emph{emergency} or a \emph{disaster}. While an emergency is an episode that requires immediate attention, but usually on a small scale, it can turn in a disaster if left unattended. For example expected heavy rainfall could lead to several emergencies (e.g.~car crashes, flooded bridges) and might eventually turn into a disaster (e.g. massive flooding or a hurricane). Disaster can be caused by nature or humans and might or might not show early signals that allow the disaster to be avoided or mitigated. Disasters have a huge impact on societies, in terms of loss of human lives and costs. Over the last 35 years, the frequency of disasters has increased five-fold and the damage caused has multiplied by approximately eight times\footnote{``Council adopts new Union Civil Protection Mechanism'' Available at: www.consilium.europa.eu/uedocs/cmsdata/docs/pressdata/en/jha/140108.pdf.}. Over the decade 1992-2012, disasters have affected 4.4 billion people and have caused USD 2 trillion USD in damage worldwide\footnote{Source: The United Nation Office for Disaster Risk Reduction (http://unisdr.org)}.

Yet, although crises are getting more frequent, people's ability to deal with adverse events, \emph{crisis management}, is also growing \autocite{Boin:2009bv}. Crisis management involves a set of collaborative inter and intra-organisation activities to respond to a crisis. Examples of typical roles and activities deployed are: police forces to constrain access to the crisis scene, firefighters to explore and map undisclosed areas, dog handlers to search for the injured, paramedics to activate triage and hospitalise the wounded; fellow citizens to report information and stay out of danger.

Activating effective crisis management strategies can avoid or reduce the extent of an emergency or a disaster, save human lives and reduce the cost of recovery. For this reason improving crisis management practices, hereafter \emph{crisis training} is a priority for many European countries \footnote{``Council adopts new Union Civil Protection Mechanism'' Available at: www.consilium.europa.eu/uedocs/cmsdata/docs/pressdata/en/jha/140108.pdf.}. Providing better crisis management is not easy task. This is due to the nature of crisis work as a complex, inter-organisational activity, often without a clear start or end, and involving many different roles.

\section{Crisis preparedness}\label{crisis-preparedness}

The effort of providing better crisis management is also known as \emph{crisis preparedness}. It is a collective activity which involves fellow citizens, crisis workers and institutions at multiple levels. Getting prepared to a crisis is a continuous process focusing on two areas, \emph{prevention} and \emph{response} \autocite{Deverell:2009fk}

\begin{figure}
	[h!] \centering 
	\includegraphics[width=1
	\textwidth]{crisis_timeline} \caption{Phases of a crisis} \label{fig:phases} 
\end{figure}

\textbf{Prevention} refers to activities aiming at avoiding a crisis; e.g.~mitigating risks by monitoring the environment and raising awareness in the population about how to recognise early warnings.

\textbf{Response} is concerned with getting ready to promptly react when a crisis occurs, in order to be able to mitigate it so that there is as little damage as possible and to reduce its impact on the population. It includes activities related to protocol formalisation and training of crisis workers.

Crisis workers are trained volunteers and professionals to provide help to the people in need; for example firefighters, police, paramedics. Training of crisis workers, hereafter \emph{crisis training}, is a critical activity to improve crisis management because it deals with the ability of people to react to a crisis and reduce the risks of the same turning into an emergency or disaster. As matter of fact, previous research has shown that the outcome of a disaster is highly correlated with preparation and training prior the beginning of the crisis \autocite{Asproth:cb}.

Different approaches and activities related to \emph{crisis training} are described in the next section.

\section{Crisis training}\label{crisis-training}

Four approaches to crisis training have been identified: protocol training, tabletop exercises, physical simulations and serious games. They share the goal to produce learning outcomes towards better crisis management practice. The presented approaches are not mutually exclusive but rather complementary. They are the expression of a trade-off between adding realism to the training experience and costs (Figure \ref{fig:training-activities}).

\begin{figure}
	[h!] \centering 
	\includegraphics[width=1
	\textwidth]{training_examples} 
	\caption{Cost and realism in different training activities} 
	\label{fig:training-activities} 
\end{figure}

\textbf{Protocol training} - It is a formal learning activity related to teaching of procedures, protocols and best practice. This is often the first type of training given to newcomers, e.g.~by means of a crisis management course.

\textbf{Tabletop exercises} - Often performed at the strategic level, it usually involves disaster managers to gather together and talk through a simulated disaster. There is usually little realism in a tabletop exercise\footnote{source: \url{http://www.epa.gov/watersecurity/tools/trainingcd/Pages/exercise-menu.html}}: equipments are not used, resources are not deployed in space and time constraints are not introduced. Tabletop exercises usually run for a few hours, the limited scale of the exercises make them a cost-effective tool to validate plans and activities.

\textbf{Physical simulations} - They are large-scale events that try to recreate as much as possible events and context from real crises, in terms of environment, tasks and challenges, stress and emotions. Simulated crisis events can run for days, they take place on-location using, as much as possible, equipment and personnel that would be deployed on a real event. Simulations involve a wide range of roles from disaster manager, to team leaders and field workers. They are high cost events and for this reason are run sporadically; it is therefore important to maximise their training outcomes.

Simulations usually take place in remote areas unaccessible to the public, which are set up to recreate harsh conditions like the presence of debris, flooded terrains, fire ashes and broken cars. In this setting, volunteers impersonating the injured to be rescued are located in places undisclosed to the trainees (Figure \ref{fig:simulation-phases}-left).
\begin{figure}
	[tbh] \centering 
	\includegraphics[width=1
	\textwidth]{simulation_phases} \caption{Different phases of a simulation, setup (left), work (centre), debriefing (right). Pictures were taken during field studies performed by the author.} \label{fig:simulation-phases} 
\end{figure}

A typical training session includes \emph{briefing}, \emph{simulation} and \emph{debriefing} phases. During briefing the esercise manager describes the settings, and assigns duties to the teams. During simulation workers implement rescue procedures (Figure \ref{fig:simulation-phases}-centre). The work involves cooperation among: police forces, to handle traffic and fence the operational area, firefighters to explore and secure undisclosed areas, civil protection workers to build field hospitals, dog handlers to search for survivors and teams of paramedics to activate triage, treat the injured and transportation to the nearest field hospital. A collaborative debriefing of the events, with focus on time of completion of procedures and issues that might have been arisen during the practice, concludes the simulation (Figure \ref{fig:simulation-phases}-right).

During this PhD work the author performed six field studies during physical simulations. A list of the studies and methods adopted is given in Chapter \ref{research}.

\textbf{Serious games} - They aim at teaching useful skills for crisis management leveraging the ``fun'' aspect of games as a motivator to play repeatedly and gain multiple perspectives \autocite{DiLoreto:2012bj}. Rather than seeking to teach \emph{hard skills} like protocols and best practices, serious games work best at enhancing \emph{soft skills} e.g.~communication styles, stress management and coping strategies \autocite{Sagun:2009ks}. Those skills are useful both for prevention and for response to a crisis. Serious games bridge the gap between tabletop exercises and physical simulations: they are more realistic than the former yet without the huge costs of the latter. They can be played multiple times, both by individuals and collaboratively by teams. An ecology of serious games can address a variety of roles and tasks: being a lightweight training tool, each game can be tailored on a specific learning objective. Serious games for crisis training range from board game to highly immersive virtual environments, for a review see \autocite{DiLoreto:2012bj}.

While the first approach presented relies on formal learning (e.g.~classroom teaching), the other approaches can benefit from \emph{experiential learning} \autocite{kolb1984organizational} techniques; being the training experience focused on doing some extent of real work.

\subsection[Experiential learning]{Experiential learning, one of the sought-after outcomes of crisis training}\label{experiential-learning-one-of-the-sought-after-outcomes-of-crisis-training}

Experiential learning is one of the sought outcomes of crisis training. It is an informal learning approach that makes use of work experience and reflection in order to achieve learning outcomes \autocite{kolb1984organizational}. Although learning by experience and lesson drawing are still quite unexplored areas of crisis management \autocites{Lagadec:1997js}{Boin:2007wt}{Stern:1997eb}, the important role of \emph{experience} in crisis training, as means for achieving organisational learning, is widely acknowledged.

Experience gathered during real and simulated crises can be used to achieve a learning outcome \autocite{Deverell:2009fk}, which may occur ``when experience systematically alters behaviour or knowledge'' \autocite[p.233]{Schwab:2007iw}. Larsson \autocite{Larsson:2010jr} highlights how past experience (e.g.~from an earlier training event) can hold knowledge useful for managing a new crisis for example to correct mistakes done in the past: ``Personal and group experiences, together with exercises, seem to be the two most important forms of learning'' \autocite[p.714]{Larsson:2010jr}. As stated by Hillyard \autocite*{Hillyard:SYlJRQLN}, ``...learning together from an event in order to prevent, lessen the severity of, or improve upon responses to future crises''. The correct action to take often can only be derived from experience, e.g.~handling of the events in similar situations.

Yet, learning from crisis work experience is not easy. While learning during a crisis, or \emph{intra-crisis}, is very difficult due to time pressure, stress and demands for rapid action \autocite{Deverell:2009fk} and because learning is typically a retrospective exercise \autocite{jasanoff1994learning}; \emph{inter-crisis} learning, before and after a crisis event, is also challenging. Moynihan \autocite*{Moynihan:2008gq} identifies ten barriers to effective crisis learning. Among those the high consequentiality of crises makes experiential learning costly \autocite{LaPorte:1991gk}, moreover the specificity of each crisis event makes hard to apply learning outcomes from one crisis to another.

\emph{Therefore, how work experience can produce a learning outcome? And what are those learning outcomes?}?

Work experience can be turned into new knowledge thanks to the reflective practice. Reflecting on action allows workers to learn from past experience with the goal of performing better in the future \autocites{boud1985reflection}{Schon:1983ut}. In addition, sensing-based interfaces can augment the reflective practice, for example providing sensors for capturing different aspects of experience, user interfaces to facilitate the practitioner in reflecting upon information captured and infrastructures to share data and reflection outcomes among practitioners. The common denominator is that technology can add realism to exercises, in order to re-create experiences that are as close as possible to real crisis, and allow trainees to experience emotions (e.g.~stress) of a similar nature and intensity as the ones experienced under a real emergency \autocite{MacKinnon:2012wz}. Adding realism to the training experience is recognised to be a key for achieving learning outcomes \autocite{Asproth:2013vs}.

In the remaining of this thesis, theoretical frameworks presented in the next chapter describe how to promote reflection by identifying relevant \emph{cycles} of learning activities. Chapter \ref{interaction} describes how those activities can be augmented by sensing-based technologies. 

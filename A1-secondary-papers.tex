\chapter{Secondary papers}
\label{secondary-papers}

In this appendix papers which are not included in the PhD thesis are briefly summarised. The papers present work-in-progress or incremental achievements that have led to the research results reported in the thesis.

Each summary includes:

\begin{itemize}
	\item Title
	\item Authors
	\item Where the paper was published
	\item Brief description of the paper's contribution
\end{itemize}

\section*{Paper 1}\label{secondary-papers}

\emph{Title: }WATCHiT: Towards wearable data collection in crisis management

\emph{Authors: }Simone Mora and Monica Divitini

\begin{quote}
	\emph{Abstract: }In this paper we present the work-in-progress on WATCHiT, a wristband computer for data collection during crisis response work. We outline the user-centered research methodology we adopt and we identify four design challenges to be tackled. We report the design of a working prototype that relies on wearable sensors to capture quantitative data and on an eyes-free, token-based interface for tagging sensor data with user-defined text messages. The prototype has been evaluated with emergency workers during a simulated rescue operation.
\end{quote}

\emph{Published in: }Work-in-progress at the Eight International Conference on Tangible, Embedded and Embodied Interaction (TEI), 2014.

\emph{Description: }The papers present a first draft of the design challenges for experience-capturing tools and a description of a early prototype of \emph{WATCHiT}. 

\section*{Paper 2}

\emph{Title: }Supporting Crisis Training with a Mobile Game System

\emph{Authors: }Ines Di Loreto, Emil Mork, Simone Mora and Monica Divitini

\begin{quote}
	\emph{Abstract: } Crisis training is highly complex and it requires multiple approaches. Games have a high potential in this context because they might support players in exploring different situations and experience different crisis scenarios. This paper proposes a mobile game system for crisis training. The system aims to promote soft skills and basic procedures learning. The system is composed by (i) a website that allows to set up the game and review game results and (ii) a mobile game. The set up supports the tailoring of games that better fit the specific learning needs of the players. The actual play promotes gaining of experience. The final review is intended to promote reflection on the gained experience, mirroring debriefing sessions that are common in crisis situations. Results from the initial evaluation show that the game and the post-game reflection are useful to train soft skills and to improve behavior.	
\end{quote}

\emph{Published in: }Proceedings of the International Conference on Serious Games Development Applications (SGDA), 2013.

\emph{Description: }The game presented in the paper is a mobile version of the \emph{Don't Panic} board game, aiming at providing alike learning objectives, yet via a collaborative pervasive gaming experience.

\section*{Paper 3}

\emph{Title: }Token-based Interaction with embedded digital information

\emph{Authors: }Simone Mora

\begin{quote}
	\emph{Abstract: }Embedding digital information into places and objects can improve collaborative processes by allowing a piece of information to travel across different contexts of use. Yet tools for supporting the processes of information embedding, discovery and visualization are needed. This PhD-work aims at providing a conceptual framework that promote the use of (in)tangible tokens to enable information embeddedness. The framework is used to drive the design of pervasive applications to support collaboration and reflection in crisis management.
\end{quote}

\emph{Published in: }Doctoral consortium of the International Conference on Tangible, Embedded and Embodied Interaction (TEI), 2013.

\emph{Description: }This paper details a work-in-progress on the research questions and methodology adopted throughout the PhD work. It also describes early contributions.

\section*{Paper 4}

\emph{Title: }Tangible and Wearable User Interfaces for Supporting Collaboration among Emergency Workers

\emph{Authors: }Daniel Cernea, Simone Mora, Alfredo Perez, Achim Ebert, Andreas Kerren, Monica Divitini, Didac Gil de La Iglesia and Nuno Otero

\begin{quote}
	\emph{Abstract: }Ensuring a constant flow of information is essential for offering quick help in different types of disasters. In the following, we report on a work-in-progress distributed, collaborative and tangible system for supporting crisis management. On one hand, field operators need devices that collect information, personal notes and sensor data, without interrupting their work. On the other hand, a disaster management system must operate in different scenarios and be available to people with different preferences, backgrounds and roles. Our work addresses these issues by introducing a multi-level collaborative system that manages real-time data flow and analysis for various rescue operators.
\end{quote}

\emph{Published in: }Proceedings of the CRIWG Conference on collaboration and technology, 2012

\emph{Description: }This paper presents the first investigation to the use of wearable sensors for data-capture during crisis work. Although the scenarios reviewed by the paper don't focus on training, a prototype of a multipurpose wearable sensor is presented and integrated with an existing system for crisis management. The prototype will be later repurposed to assist crisis training scenarios.

\section*{Paper 5}

\emph{Title: }Collaborative Serious Games for Crisis Management: An Overview

\emph{Authors: }Ines di Loreto, Simone Mora and Monica Divitini

\begin{quote}
	\emph{Abstract: }Training in the field of crisis management is complex and costly, requiring a combination of approaches and techniques to acquire not only technical skills, but also to develop the capability to cooperate and coordinate individual activities towards a collective effort (soft skills). In this paper we focus on serious games for increasing participants’ skills in a playful manner. In the paper we identify general issues characterizing crises management and we analyze the state of the art of serious games for crisis management in order to understand strengths and weaknesses of these environments.
\end{quote}

\emph{Published in: }IEEE International Workshop on Enabling Technologies: Infrastructure for Collaborative Enterprises (WETICE), 2012

\emph{Description: }The paper presents a literature review in the field of serious games for supporting crisis training. The paper also provides a set design suggestions that will be addressed with the development of \emph{Don't Panic}

\section*{Paper 6}

\emph{Title: }Mobile and Collaborative Timelines for Reflection

\emph{Authors: }Anders Kristiansen, Andreas Storlien, Simone Mora, Birgit R. Krogstie and Monica Divitini

\begin{quote}
	\emph{Abstract: }In this paper we present the design and evaluation of TimeLine, a mobile application to support reflective learning through timelines. The application, running on Android devices, allows users to capture traces of working and learning experiences in a timeline with the aim to provide data that can be used to promote reflection and learning after the experience. The paper presents the design of the application, its evaluation, and identifies challenges connected to the development and deployment of timelines for reflection.
	
\end{quote}

\emph{Published in: }Proceedings of the IADIS International Conference Mobile Learning. 

\emph{Description: }The paper proses the use of timelines to capture and visualise traces of working experiences, with the goal to promote reflective learning. Based on the work in this paper, the use of the timelines will be later integrated with augmented reality approach in the design of a mobile app, \emph{CroMAR}, to support in-situ debriefing after crisis work.

\section*{Paper 7}

\emph{Title: }Supporting Mood Awareness in Collaborative Settings

\emph{Authors: }Simone Mora, Veronica Rivera-Pelayo and Lars Müller

\begin{quote}
	\emph{Abstract: }Affective aspects during collaboration can be exploited as triggers for reflection, yet current tools usually ignore these aspects. In this paper, we present a set of design choices to inform the design of systems towards enabling mood awareness in collaborative work settings like meetings or conferences. Design choices have served to outline and implement a collection of prototypes, including two input interfaces and three visualizations, which have been evaluated during an important project meeting over three days. Our results show that (a) the user acceptance of capturing mood is high (b) the aggregated mood values are related to the work process and (c) aggregated moods can influence the individual by creating awareness of others. Further discussion on the impact of the different design choices shows promising venues to improve mood awareness support.
	
\end{quote}

\emph{Published in: }Proceedings of the International Conference on Collaborative Computing (CollaborateCom), 2011

\emph{Description: }This paper investigates the capture and visualisation of moods and emotions as triggers for reflection. Keeping track of emotions experienced by workers is critical in crisis management and training. Supports for capturing moods it has been further integrated in \emph{WATCHiT}.


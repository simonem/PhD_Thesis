\chapter[Theoretical underpinning:\\ Computer Supported Reflective Learning][Computer Supported Reflective Learning]{Theoretical underpinning: Computer Supported Reflective Learning}\label{csrl}

In order to guide the design of technology to support reflective learning in crisis training, I adopted the Computer Supported Reflective Learning model (hereafter CSRL model) developed by the MIRROR project. The model identifies the requirements to design technology to support reflective learning \autocite{Krogstie:2013kf}. The CSRL model has worked as theoretical underpinning for the development of sensing-based technologies presented in this PhD work, providing a language for guiding the understanding of reflection and drafting requirements for the technology.

After a brief introduction about theories in the field of reflective learning, I describe the CSRL model and how it can be applied to the development of technology. In the following I will use the terms \emph{reflective learning} and \emph{reflection} as synonyms.

\section{The reflective practice}\label{reflection}

Boud \autocite*{boud1985reflection} defines reflective learning as ``a generic term for those intellectual and affective activities in which individuals {[}\ldots{}{]} explore their experiences in order to lead to new understandings and appreciations'', it is both an individual and collective mental process that turns past experiences into new knowledge. This is also in line with the work of Sch\"on \autocite*{Schon:1983ut} who further distinguishes between \emph{reflection-in-action} and \emph{reflection-on-action}.

Reflection consists of a three-steps process during which the learner re-evaluates her experiences inspecting behaviours, ideas and feelings; eventually deriving conclusions and lessons learned to that guide future behaviour (Figure \ref{fig:boud-model}). The process can be iterated multiple times and might influence the learner's behaviour only in the long term.

\begin{figure}
	[tbh] \centering 
	\includegraphics[width=1
	\textwidth]{boud} \caption{The reflection process according with Boud. Figure adapted from \protect\autocite{boud1985reflection}} \label{fig:boud-model} 
\end{figure}

A key aspect in making a reflective process to happen is the presence of triggers. Triggers are unexpected situations, for example disturbances and perception of uncertainty; but also positive situations like a surprising success. In general, reflection seems to be triggered by awareness of the discrepancy between expectations and the current experience. Reflection might be triggered by an external event or agent (external trigger/accident) or might develop from one's own thinking of a whole series of occurrences over time (internal trigger). Reflection can occur incidentally or intentionally, but in both cases it is a conscious evaluation of an experience. Furthermore people can learn not only from their own experiences, but also from other's experiences directly or indirectly (for example by observing and reflecting on other's actions).

Similar to the work of Boud, Kolb describes experiential learning as a cyclic process named ``The Kolb Cycle'' (Figure \ref{fig:kolb-model}).

\begin{figure}
	[tbh] \centering 
	\includegraphics[width=0.8\textwidth]{kolb} 
	\caption{"The Kolb cycle", a model of experiential learning. Figure adapted from \protect\autocite{kolb1984organizational}} \label{fig:kolb-model} 
\end{figure}

According with Kolb \autocite*{kolb1984experiential} reflection is a process that involves not only reinterpreting existing experiences, but also initial perception and interpretation of the raw experience.

For a description of other existing theories in reflective learning see \autocite{WoodDaudelin199636}.

\subsection{Collaborative reflection during debriefings}\label{debriefing-crisis-management-work-an-example-of-collaborative-reflection}

An example of collaborative reflection in crisis management is \emph{debriefing}. As outlined by Boud et al. \autocite*{boud1985reflection} debriefing is a form of collaborative reflection because during debriefings a re-evaluation of experience takes place, with explicit attention to emotions, ideas and behaviour.

Debriefing involves ``reviewing a difficult episode from a constructive point of view \ldots{} the goal is to extract fundamental lessons learned from the way the event was handled'' \autocite{Lagadec:1997js}. It is a collaborative activity involving multiple roles and it is usually performed after a (real or simulated) crisis work experience. 

Figure \ref{fig:debriefing-example} shows one of the debriefing observed during the user studies reported in Chapter \ref{research}. After a 3-day physical simulation of crisis management operations, the chief manager discusses with team leaders and field workers what went wrong and how to avoid the same issues in the future. Technology is used to visualise the location of operations on a digital map. Data were previously manually entered during the training event.

\begin{figure}
	[tbp] \centering 
	\includegraphics[width=1
	\textwidth]{debriefing} \caption{A debriefing after a physical simulation of crisis management work observed by the author} \label{fig:debriefing-example} 
\end{figure}

The outcome which debriefing seeks to obtain is lesson drawing. Previous work experience provides a good source of lesson-drawing which may potentially affect managing, planning and training for future crises. Yet lessons-drawing is often one of the most neglected aspects of crisis management \autocites{Lagadec:1997js}{Stern:1997eb}. The introduction of debriefing into crisis organisations often meets resistance \autocite{Lagadec:1997js}. This might be due to lack of commitment, costs, but also the lack of technologies to make the debriefing more effective.

\section{Computer Supported Reflective Learning (CSRL), a model}\label{computer-supported-reflective-learning-a-model}

Building on the presented theories and on empirical studies, the MIRROR project has iteratively developed a model for Computer Supported Reflective Learning (CSRL model). The model has been designed to identify requirement, design and implement technology to support for reflective learning \autocite{Krogstie:2013kf}. Rather than providing formal guidelines or pre-defined processes, the model helps to understand and analyse reflection in the workplace and it suggests how technology can support reflective practice.

Following the work of Boud et al. \autocite*{boud1985reflection} the model considers reflective learning as ``the conscious re-evaluation of experience for the purposes of guiding future behaviour {[}\ldots{}{]} as reflection transforms experience from work into knowledge applicable to the challenges of daily work'' \autocite{Krogstie:2013kf}. The model specifically addresses reflection in the workplace with \emph{work} and \emph{reflection on-action} as loosely coupled activities that have an impact on personal, collaborative and organisational growth. Therefore the model is well suited to address reflection in the crisis domain in which unexpected adverse events do not allow to schedule clear boundaries between the time to be dedicated to work and to learning.

According with the model, a \emph{reflection session} is a time-limited practice in which reflection happens. Reflection is driven by learning objectives that might be only partially explicated, leaving rooms for open-ended outcomes. Such outcomes may include a change in behaviour, new perspectives and commitment for action \autocite{boud1985reflection}. \emph{Participants} of the session might be a single person (individual reflection) or multiple persons (collaborative reflection).

The model explains reflective learning as a cycle involving four stages of reflection: (i) do work; (ii) initiate reflection session; (iii) conduct reflection session; and (iv) apply reflection outcomes. For each stage the framework specifies relevant sub-steps: specific reflection-useful activities that can be augmented with technology. For example, initiate reflection session includes \emph{decide to reflect} and \emph{frame the reflection session}.
\begin{figure}
	[ptb] \centering 
	\includegraphics[width=1
	\textwidth]{CSRL} \caption{CSRL reflection cycle. Figure adapted from \protect\autocite{Krogstie:2013kf}} \label{fig:csrl-model} 
\end{figure}

Figure \ref{fig:csrl-model} depicts the models in terms of \emph{stages}, \emph{inputs} and \emph{triggers}. A \emph{stage} includes sub-activities that can be supported with technology, \emph{inputs} are either raw or more or less contextualised data being exchanged among stages; \emph{triggers} are either external events or internal mental processes that initiate a reflection session. Reflection can be triggered during work, while a change is about to be applied, or during the reflection session itself. In general, reflection seems to be triggered by awareness of discrepancy between expectations and the current experience.

Triggers also allow for including more actors in the reflection process, iteratively starting a new cycle based on the results of previous ones. For instance, the outcome of a personal consideration (e.g.~how a crisis procedure is applied) might be brought in a team meeting to trigger collaborative reflection, ultimately leading to a change in protocols. In this way, we can look at reflection as a storyline that might involve different actors within the organisation \autocite{PrPK13}.

\section{CSRL applied to crisis training}\label{csrl-crisis}

The CSRL model can be used by designers to choose which technology to use to support reflection activities or do derive requirements for the design of new technologies. For each stage, the CSRL model identifies support that can be provided through technology. For example in the \emph{do work} phase, technology could be used to monitor work and collect data that can be useful for reflection, in \emph{initiate reflection} to set the objectives for reflection or to involve others in the session, in \emph{conduct reflection} to share work experience with colleagues; and in \emph{apply reflection outcomes} to decide how the change to work will be implemented.

The model has driven the development of several software and hardware applications within the MIRROR project; to address reflective learning in the fields of social care, health care, business and emergency aid. For a description of the applications see \autocite{Schwantzer:2014we}. 

In the case of crisis training I identified that the mapping between the activities described by the CSRL model and technology can be placed in three macro-areas (Figure \ref{fig:model-instantiation}):

\begin{figure}
	[ptb] \centering 
	\includegraphics[width=1
	\textwidth]{model-instantiation} \caption{Instantiation of the CSRL model to support crisis training} \label{fig:model-instantiation} 
\end{figure}

\begin{itemize}
	\itemsep1pt\parskip0pt\parsep0pt 
	\item technology to \textbf{capture} work experiences (for example by means of automatic sensors, a personal diary application, or a timeline visualisation) 
	\item technology to \textbf{re-create} work experiences, making use of the captured data to trigger and assist a reflection session with relevant information (for example by allowing to re-evaluate a past experience from multiple point of views, in a context that helps making sense processes during debriefing) 
	\item technology to \textbf{generate} new, realistic work experiences for training purposes (for example via virtual worlds, serious games, or tabletop exercises) 
\end{itemize}

The three areas have the common need for innovative interfaces between people and technology. Yet the design of such user interfaces aims at different goals. During \emph{experience capture} the interface should allow the collection of a variety of quantitative data and user-submitted information without interrupting crisis work. The stage of \emph{re-creating} experiences need tools for data visualisation and manipulation capable to re-create a work experiences in a context that promote reflection. Finally \emph{generating} experiences needs interfaces to bring realism and engagement of real crises into a simulated environment. Notably, the \emph{capture} of experiences is done during the work, which is subject to strict crisis protocols that limit the design space for the technology.

Although specific activities from each of the four stages have been considered, the main focus of this investigation is on supporting with technology the stages of \emph{plan and do work} and \emph{conduct reflection session}, since these stages involve activities observed during field studies.



In the following chapter I will investigate how recent advances in the field of sensing-based interaction can provide theoretical tools from human-computer interaction theory for the design of user interfaces to capture, re-create and generate crisis work experience. 

\mainmatter

\chapter{Introduction}\label{introduction}

\section{Problem statement}\label{problem-statement}

This thesis is about how to improve crisis training with ICT systems. Crisis training is an umbrella term for complex, collaborative activities aiming at improving people's ability for \emph{preparedness} to human and natural-caused crises (e.g.~a flood or a terrorist attack) \autocite{Lagadec:1997js}. The term \emph{crisis} refers to a sequence of problematic events that, if left unattended, might eventually lead to a \emph{disaster}, causing huge damage and loss of lives. Disasters have a huge impact on societies, both in terms of human beings and costs. Over the last 35 years, the frequency of disasters has increased five-fold; the damage caused has multiplied approximately eight times\footnote{Source: ``Council adopts new Union Civil Protection Mechanism'' \\ Available at: \url{www.consilium.europa.eu/uedocs/cmsdata/docs/pressdata/en/jha/140108.pdf}}. According with the United Nation Office for Disaster Risk Reduction, over the decade 1992-2012, disasters have affected 4.4 billion people and have caused USD 2 trillion in damage worldwide\footnote{Source: The United Nation Office for Disaster Risk Reduction (\url{http://unisdr.org)}}. For this reason training for better crisis management is a priority for many countries, including those in Europe.

Crisis training focuses on teaching emergency workers (e.g.~firefighters, paramedics) how to efficiently respond to a crisis; for example by actuating coping strategies and implementing rescue procedures (crisis management). Each crisis is likely to be a specific, unpredictable event that will not take place again under the same circumstances. Training for crisis preparedness is a wicked problem, however better crisis management can positively affect everyone's lives.

There are four main approaches to crisis training: protocol training, tabletop exercises, physical simulations and serious games; on top of that real crises also offer triggers for learning \autocite{Deverell:2009fk}. In this research work we look at medium to large scale physical simulations and serious games as key training practices; aiming at advancing them with technology. Physical simulations recreate at best real crises in terms of environment (e.g. presence of debris, collapsed buildings), and in reproducing feelings experienced by crisis workers such as stress, tension, time pressure and uncertainty \autocite{Borodzicz:2002em}. Yet they are determinated by high set-up cost and the large effort required to coordinate multiple organisations and dozens of field workers. Moreover it has been observed that the impact of those events is limited by lack of technologies, for example to capture data and maintain an overview of rescue efforts \autocite{Kyng:2006he}. Otherwise, serious games trade the realism of physical simulations to provide a more lightweight training experience which can be easily reproduced frequently by single workers or teams. The ``fun'' element typical of games is added as motivator to engage workers in frequent play. In this perspective physical simulation and serious games are are not mutually exclusive but rather complementary approaches to crisis training, providing realistic training experiences.

\emph{The research in this thesis investigates how to maximise crisis training learning outcomes during physical and serious game-based simulation of crisis work. It is based on the assumption that training practices already in place can be enhanced by combining (i) reflective, experience-based learning approaches and (ii) advances in ICT and sensing-based interaction \autocite{Zhai:2005jm}.}\todo{Reforumlate this sentence}

Experience-based learning is a powerful tool. Facilitating learning from work experience of the different roles in the field (e.g. disaster managers vs. field workers) can bring outcomes to complement traditional formal training. Learning from experience entails reflection \autocites{boud1985reflection}{Dewey:1998ug}{kolb1974toward}. Reflection on action has been a research topic since the work of Dewey \autocite{dewey1933we} that describes how we learn by comparing our expectations to new and past experience. Reflecting on action is critical in order to learn from past experience with the goal of performing better in the future \autocites{boud1985reflection}{Schon:1983ut}; and a number of tools have been developed to support reflection, as an individual or collaborative activity. Among those, the CSRL (Computer Supported Reflective Learning) model developed as part of the MIRROR project\footnote{MIRROR Project - \url{www.mirror-project.eu}} aims at providing guidelines to develop technology tools to support reflection. It identifies a cycle of four stages of reflection \autocite{Krogstie:2013kf}: \emph{do work}, \emph{initiate reflection session}, \emph{conduct reflection session} and \emph{apply reflection outcomes}. For each stage a number of reflection-useful activities that can be augmented with technology are provided.

In the context of crisis training, reflection activities can be summarised in three areas: (i) capturing work experience, (ii) re-creating work experience and (iii) generating realistic work experience. Technology provides help in different ways. Sensors can capture aspects of real or simulated work experience, including qualitative and quantitative elements; data which can be visualised on a interactive computer interface to provide triggers for re-evaluating an experience towards a learning outcome, or that can be used to plan new training work. Yet current ICT tools do not consider the very specific, situated nature of crisis work. While data capturing tools lack interaction paradigms suitable for being used during work, visualisation tools struggle in providing the user with the context information needed to ground reflection on past work experiences and to achieve learning outcomes that are structured to be easily shared among colleagues. Moreover the introduction of technology is impeded by resistances in organisations that might be reluctant to modify accustomed practices, even if unproductive \autocite{JCCM:JCCM15}.

Theories in the field of \emph{sensing-based interaction}, can inform the design of novel technologies to better assist reflective learning in crisis training. Sensing-based interaction is a trend in HCI which promotes sensing information to make human-computer interfaces, sensing-based interfaces, more effective \autocite{Zhai:2005jm}. \emph{Tangible} and \emph{embodied} \autocite{Dourish:2001vc} are two characterising traits of such interfaces. They aim at enabling interaction with digital information exploiting the affordances that everyday objects provide, rather than traditional paradigms such as keyboard, mouse or touchscreens. Using sensor-based technology, conventional objects can be augmented and turned into ``physical handles'' for digital operations \autocite{Ishii:1997ur}, linking their traditional affordances to new digital meanings. Making interaction with computers more ``physical'' allows for leveraging humans' skills for interaction with the real world \autocite{Shaer:2009fx}. This approach might be well suited for crisis fieldwork which, contrarily to traditional office work, has a strong physical and spatial connotation. In this perspective, tangible and embodied interfaces have been successfully employed to provide natural \autocite{Terrenghi:2005gq} and situated \autocite{Klemmer:2006ez} learning and increased reflection and engagement \autocite{Rogers:2006te}. ``Being able to move around in the world and interact with pieces of the world enables learning in ways that reading books and listening to words do not''. \autocite{Klemmer:2006ez}

Yet building prototypes of sensing-based interfaces, is a complex task which requires scientists to master a wide set of skills including product design, hybrid software and electronics development, as well as hardware construction and assembly. This is still a relatively new area in HCI and it is characterised by the absence of a widely established toolchain to help the prototyping work. Rather, it is characterised by fast adoption of edging technologies and a pragmatic attitude at \emph{tinkering} and \emph{thinking-thrugh-prototyping} \autocite{Klemmer:2006ez}. Considering the essential role of prototypes in the development of novel technology; developing a skillset to enable rapid prototyping of hybrid software/hardware artefacts is essential to the accomplishment of the goals sought by this PhD work.

\begin{figure}
	[tbh] \centering 
	\includegraphics[width=0.70
	\textwidth]{better_management} \caption{Relation among the three domain of this thesis} \label{fig:topic_relation} 
\end{figure}

\section{Research methodology}\label{research-methodology}

The work in this thesis is based on \emph{design research} \autocites{Hevner:2010gc}{March:1995gm}. The work followed a \emph{user centred approach} \autocites{MAGUIRE:2001dp}{Gulliksen:2003hd}, based on exploratory studies and design work in multiple iterations.

Several qualitative research methods \autocite{robson1993real} have been adopted, including shadowing and observation of crisis worker during field studies. Scenarios, personas and mockups aided the user-centred design work. Consistently with design research methodology, grounded on the activities of \emph{building} artefacts for a specific purpose and of \emph{evaluating} how well the artefacts perform \autocite{March:1995gm}, a number of prototyping iterations and evaluation studies have been performed.

Prototyping involved the construction of sensing-based interfaces to support reflection processes. The design of prototypes was grounded in field studies during physical simulations of crisis that I attended. Simple prototypes were initially used to build a deeper understanding of the crisis domain, where I did not have any previous knowledge. They acted as technology probes \autocite{Hutchinson:2003il} and facilitated building and understanding of the crisis domain by engaging users in focus groups. Later, multiple iterations implemented a growing set of requirements in fully working prototypes that were robust enough to be deployed during simulated crisis work.

User evaluations followed each design iteration (Figure \ref{fig:cromar}). The aim was both to assess usability of the prototypes and impact on reflection outcomes. Prototypes were evaluated both during focus groups and during large simulations of crisis response work. Results from evaluations have fed the following design iterations, and contributed in the validation of theories on reflective learning and into the development of new constructs.
\begin{figure}
	[tbh] \centering 
	\includegraphics[width=1\textwidth]{introduction_cromar} 
	\caption{One of the evaluation field studies performed} 
	\label{fig:cromar} 
\end{figure}

\section{Research questions}\label{research-questions}

The main research question for the PhD work is:
\begin{quote}
	MRQ: \MRQ 
\end{quote}

To answer the main research question the work has been broken down into three sub-questions:
\begin{quote}
	RQ1: \RQi 
\end{quote}
\begin{quote}
	RQ2: \RQii 
\end{quote}
\begin{quote}
	RQ3: \RQiii 
\end{quote}

While the first two questions aim at investigating the design of systems to provide technology support for the tasks of \emph{capturing}, \emph{re-creating} and \emph{generating} work experience; the third question investigates how toolkits and open-source communities can ease the implementation of design ideas into prototypes.

\section{Research outcomes}\label{research-outcomes}

There are three main outcomes for this PhD work.

Seven research papers published in peer-reviewed conferences and journals explored the research questions.

Building on results reported in the papers, a body of knowledge contributing in the fields of Technology Enhanced Learning (TEL), Information Systems for Crisis Response (ISCRAM) and Tangible, Embedded and Embodied Interaction (TEI) has been developed.

Finally research contributions have been evaluated for commercial exploitation. Five \emph{Disclosure of Invention (DOFI)} have been filed for technology transfer and early contacts with the industry have been established.

\subsection{Research papers}\label{research-papers}

The research questions RQ1-RQ3 are addressed in the following research papers:
\begin{quote}
	\textbf{P1:} Mora, S., Boron, A., \& Divitini, M. (2012). CroMAR: Mobile Augmented Reality for Supporting Reflection on Crowd Management. \emph{International Journal of Mobile Human Computer Interaction}, 4(2), 88--101. 
\end{quote}
\begin{quote}
	\textbf{P2:} Mora, S., \& Divitini, M. (2014). Supporting Debriefing with Sensor Data: A Reflective Approach to Crisis Training. \emph{In Proceedings of Information Systems for Crisis Response and Management in Mediterranean Countries, ISCRAM-MED}, 196(7), 71--84. 
\end{quote}
\begin{quote}
	\textbf{P3:} Mora, S., \& Divitini, M. (2014). WATCHiT: a modular and wearable tool for data collection in crisis management and training. \emph{In Proceedings of the European Conference in Ambient Intelligence, AMI}, 8850(22), 274-289. 
\end{quote}
\begin{quote}
	\textbf{P4:} Di Loreto, I., Mora, S., \& Divitini, M. (2012). Don't Panic: Enhancing Soft Skills for Civil Protection Workers. \emph{In Proceedings of International Conference on Serious Games Development Applications, SGDA}, 7528(1), 1--12. 
\end{quote}
\begin{quote}
	\textbf{P5:} Mora, S., Di Loreto, I., \& Divitini, M. The interactive-token approach to board games. \emph{Ready for submission}. 
\end{quote}
\begin{quote}
	\textbf{P6:} Müller, L., Divitini, M., Mora, S., Rivera-Pelayo, V., \& Stork, W. (2014). Context Becomes Content: Sensor Data for Computer Supported Reflective Learning. \emph{IEEE Transactions on Learning Technologies}, PP(99). 
\end{quote}
\begin{quote}
	\textbf{P7:} Mora, S., \& Farshchian, B. A. (2010). A Unified Architecture for Supporting Direct Tag-Based and Indirect Network-Based Resource Discovery. \emph{In Proceedings of the International Conference on Ambient Intelligence, AMI}, 6439(20), 197--206. 
\end{quote}

Table \ref{tab:rq-papers-relation} shows the mapping between research papers and research questions. In addition to these papers, this PhD work has produced seven secondary conference papers. These works present incremental achievements in research that have added to the investigation of the research questions. Abstracts of the papers are included in Appendix \ref{secondary-papers}.

\begin{table}
	[tbh] \centering \caption{The relation between research papers and research questions} \label{tab:rq-papers-relation} 
	\begin{tabular}
		{P{\dimexpr 0.075\linewidth-2\tabcolsep}P{\dimexpr 0.40\linewidth-2\tabcolsep}ccccccc} \toprule \multicolumn{2}{l}{Research questions} & P1 & P2 & P3 & P4 & P5 & P6 & P7 \\
		\midrule RQ1 & \RQi & & \textbullet & \textbullet & & & \textbullet & \\
		RQ2 & \RQii & \textbullet & \textbullet & & \textbullet & \textbullet & \textbullet & \\
		RQ3 & \RQiii & & & \textbullet & & \textbullet & & \textbullet \\
		\bottomrule 
	\end{tabular}
\end{table}

\subsection{Research contributions}\label{research-contributions}

The seven papers published added to the following contributions. 

\begin{quote}
	\emph{\textbf{C1:} \Ci.} It includes a validation of previous theoretical models and the formulation of new constructs. 
\end{quote}
\begin{quote}
	\emph{\textbf{C2:} Knowledge about designing experience-capturing tools for crisis workers.} It defines the design space as well as design challenges for building computer-based data capturing tools. 
\end{quote}
\begin{quote}
	\emph{\textbf{C3:} Novel sensing-based interaction techniques to support re-creation and generation of work experiences in crisis training.} It describes novel sensing-based interfaces for the visualisation and manipulation of data captured from work experience. 
\end{quote}
\begin{quote}
	\emph{\textbf{C4:} Knowledge about implementing prototypes to be deployed into the wild.} It presents challenges and lessons learnt derived from the author's experience in building prototypes of sensing-based interfaces. 
\end{quote}

\begin{figure}
	[!p] \centering 
	\includegraphics[width=1
	\textwidth]{papers-contributions-mapping} \caption{Research papers and the main topics of the research contributions} \label{fig:mapping} 
\end{figure}

The relation of research papers with respect to the research contributions and communities is represented in Figure \ref{fig:mapping}

\subsection{Towards exploitation of research}\label{exploitation-of-research-contributions}

During the final phase of the investigation, commercial exploitation of research contributions has been investigated. The focus was on assessing efforts needed and path of actions to evolve the prototypes developed as theory demonstrator into commercial products. To this end, I co-authored five \emph{Disclosure of Invention} (DOFI): technical documents that capture the description of the technologies created and establish inventorship. DOFIs were drafted based on information published in research papers (Table \ref{tab:papers-inventions})

\begin{table}
	[tbh] \centering \caption{The relation between registered inventions and research papers} \label{tab:papers-inventions} 
	\begin{tabular}
		{P{\dimexpr 0.05\linewidth-2\tabcolsep}P{\dimexpr 0.30\linewidth-2\tabcolsep}P{\dimexpr 0.50\linewidth-2\tabcolsep}P{\dimexpr 0.15\linewidth-2\tabcolsep}} \toprule & Authors & Invention & Research papers \\
		\midrule I1 & Mora, S., Boron, A. and Divitini, M. & CroMAR. Situated reflection and training in crisis management & P1, P2 \\
		\noalign{\smallskip} I2 & Mora, A. and Divitini, M. & WATCHiT. Wearable data collection in crisis management and training & P2, P3 \\
		\noalign{\smallskip} I3 & Di Loreto, I., Mora, S. and Divitini, M. & ``Don’t Panic!'' A serious game for enhancing soft skills for Civil Protection workers & P4, P5 \\
		\noalign{\smallskip} I4 & Mora, S., Di Loreto, I. and Divitini, N. & Anyboard: a platform for creating and play digital board games & P5 \\
		\noalign{\smallskip} I5 & Mora, S. and Divitini, M. & TILES Toolkit. Building seamless interfaces between people and the Internet of Things & P3, P5 \\
		\bottomrule 
	\end{tabular}
\end{table}

Disclosure of inventions were filed at the NTNU Technology Transfer Office\footnote{NTNU Technology Transfer AS - \url{www.tto.ntnu.no}}, a business incubator affiliated with NTNU, and in accordance with Norwegian law\footnote{In accordance with ``Act respecting the right to employees' inventions 17.4-1970'', and NTNU's internal Guidelines for innovation}. They were used by technology transfer managers to assess patent applicability and establishment of commercial activities. To this effort, I presented the research outcomes to several representatives from the industries working in the emergency management field, raising positive and supportive feedbacks. In November 2014 I was granted by NTNU Discovery \footnote{NTNU Discovery - \url{http://ntnudiscovery.no}} a NOK 150.000 (about USD 22.000) seed fund for financing further commercial exploration of the research results after the PhD completion.

\section{Context of the work}\label{context-of-the-work}

The research presented in this thesis is framed within the EU-funded (IST-FP7) project MIRROR\footnote{MIRROR Project - \url{www.mirror-project.eu}}. The objective of MIRROR is to empower and engage employees to reflect on past work performance and personal learning experience in order to learn in “real-time” and to creatively solve pressing problems. MIRROR is to help employees to increase their level and breadth of experience significantly within a short time by capturing the experience of others.  

As an associate researcher of MIRROR I took part in shaping the results of the projects by designing and implementing ICT systems, writing deliverables and attending project meetings. Thanks to MIRROR I cooperated with crisis worker associations to run field studies. I also benefited from discussions, joined works and co-authored publications with members of the consortium. After the project final review in September 2014, MIRROR has been graded as ``Excellent'' by the EU Commission.

During the PhD I was a visiting fellow to two foreign institutions: \emph{City London University}\footnote{City London University - \url{http://city.ac.uk}} in London (UK), where I was supervised by professor Neil Maiden; and \emph{MIT SENSEable City Lab}\footnote{MIT SENSEable City Laboratory - \url{http://senseable.mit.edu}} in Cambridge, MA (USA), under the supervision of professor Carlo Ratti. The purpose of the two visits was to investigate whether the technologies developed during the PhD could be generalised to application domains outside crisis training. A summary of the activities performed as visiting fellow is provided in Appendix \ref{abroad}.

I also co-advised the thesis work of eight master's students who have contributed to the development of prototypes. One of them co-authored P1.

\section{Structure of the thesis}\label{structure-of-the-thesis}

The thesis is composed by two parts:

\begin{itemize}
	\item \textbf{Part I} presents the introduction to this work. It gives on overview of the background, the methods used, the results achieved and the contributions provided by the thesis.
	\item \textbf{Part II} contains the seven research papers that added to the results of this thesis
\end{itemize}

The rest of \textbf{Part I} is organised as follows:

\begin{itemize}
	\item \textbf{Chapter \ref{crisis}} introduces the crisis domain providing an overview on scenarios, activities and roles; and presenting debriefing as a tool for reflective learning.

	\item \textbf{Chapter \ref{csrl}} describes relevant background theory on reflective and experience-based learning with focus on describing the Computer Supported Reflective Learning model adopted as theoretical underpinning of this research work.

	\item \textbf{Chapter \ref{interaction}} presents relevant background theory in sensing-based interaction, motivating the use of that paradigm applied to reflective learning.

	\item \textbf{Chapter \ref{research}} depicts the research strategy and approach adopted by this PhD work, giving overview of the user studies conducted and prototypes built.

	\item \textbf{Chapter \ref{results}} summarises the results for the research papers.

	\item \textbf{Chapter \ref{contributions}} outlines the contribution of this thesis and their relations to the research papers.

	\item \textbf{Chapter \ref{evaluation}} proposes an evaluation of the work done.

	\item \textbf{Chapter \ref{conclusions}} concludes the thesis and sketches out future research and innovation work.

	\item \textbf{Appendix \ref{secondary-papers}} summarises secondary research papers that were written during the research fellowship.

	\item \textbf{Appendix \ref{abroad}} outlines research done during academic visits in foreign institutions.

	\item \textbf{Appendix \ref{toolkits}} includes a benchmark of hardware toolkits for rapid prototyping which has been used to select the specific tools used to implement the prototypes in this PhD. 
\end{itemize}

\textbf{Part II} contains the seven research papers in full length.

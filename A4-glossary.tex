\chapter{Glossary} \label{glossary}

\textbf{Crises} are sequences of alarming events that place communities in danger, affecting millions of people worldwide. Marked by a mix of foreseeable and unexpected circumstances, a crisis can ultimately lead to a disastrous event (e.g. an hurricane or a nuclear explosion) causing huge damage and loss of human lives.

\textbf{Crisis management}

\textbf{Crisis preparedness}

\textbf{Crisis training}

\textbf{Crisis work}

\textbf{Disaster}

\textbf{Emergency}

While an emergency is an episode that requires immediate attention, but usually on a small scale, it can turn in a disaster if left unattended. For example expected heavy rainfall could lead to several emergencies (e.g.~car crashes, flooded bridges) and might eventually turn into a disaster (e.g. massive flooding or a hurricane).

Crisis management involves a set of collaborative inter and intra-organisation activities to respond to a crisis. Examples of typical roles and activities deployed are: police forces to constrain access to the crisis scene, firefighters to explore and map undisclosed areas, dog handlers to search for the injured, paramedics to activate triage and hospitalise the wounded; fellow citizens to report information and stay out of danger.

Crisis workers are trained volunteers and professionals to provide help to the people in need; for example firefighters, police, paramedics.